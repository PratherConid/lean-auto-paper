\section{Motivating Examples} \label{motex}

In section \ref{exabst} and \ref{exinst}, we will demonstrate the execution of the
quantifier instantiation and $\lambda_\to^*$ abstraction algorithms of our monomorphization procedure on a
concrete example. Note that both the example and the algorithms
here are simplified. Sect. \ref{exqdet} discusses the challenges posed by dependent
type theory and Lean4.

\subsection{$\lambda_\to^*$ Abstraction} \label{exabst}

\begin{figure}
  \begin{CenteredBox}
    \lstinputlisting[style=leanHH]{LeanCode/reverse_map_pretty.inp}
  \end{CenteredBox}
  \caption{Lean4 proof state of a problem involving \textbf{List}} \label{leanlistpretty}
\end{figure}

The Lean4 proof state of the problem we will consider is shown in Figure \ref{leanlistpretty}.
The hypotheses are displayed before $\vdash$, while the goal comes after $\vdash$.
\vmaprev states that \vmap commutes with \vrev, and
\vrevrev states that \vrev is the inverse function of itself.

\begin{figure}
  \begin{CenteredBox}
    \lstinputlisting[style=leanHH]{LeanCode/reverse_map_explicit.inp}
  \end{CenteredBox}
  \caption{Lean4 proof state after quantifier introduction and proof by contradiction, with implicit arguments displayed.
    Note that the equality sign in Figure \ref{leanlistpretty} is a syntactic sugar of
    the polymorphic function \vEq seen here}
  \label{leanlistexplicit}
\end{figure}

Since the problem is already in the $\lambda C$ fragment of Lean, the only
preprocessing step involved here is introducing all the $\forall$ quantifiers
of the goal into the context, and applying proof by contradiction. The resulting proof state
is shown in Figure \ref{leanlistexplicit}.
For clarity, we have displayed the implicit arguments of all the functions.

First, we focus on translating \vneggoal into $\text{HOL}^*$. Following the
discussion of Sect. \ref{subencmon}, we would like to find an $\text{HOL}^*$ formula $\varphi$
and a ``substitution'' $\sigma$ such that the image of $\varphi$ under $\sigma$ is \vneggoal.
We also want the problem to be provable after the translation, so $\varphi$
should preserve as much information in \vneggoal as possible.

Three polymorphic functions: \vEq, \vmap and \vrev, occur in \vneggoal.
Although these functions are polymorphic, instances of these functions
with dependent arguments instantiated could behave like $\text{HOL}^*$ variables
(we will refer to such instances as $\mathit{HOL}^*$ \textit{instances}).
The type constructor \texttt{List} is also not allowed in $\text{HOL}^*$, but
\texttt{List A} and \texttt{List B} behave just like $\text{HOL}^*$ type variables
(we will refer to expressions such as \texttt{List A} and \texttt{List B} as $\mathit{HOL}^*$ \textit{type instances}).
Therefore, we can choose
$$\begin{aligned}
  \varphi := & \ \neg (\mathsf{EqLB} \ (\mathsf{rB} \ (\mathsf{mAB} \ f^* \ (\mathsf{rA} \ \mathit{xs}^*))) \ (\mathsf{mAB} \ f^* \ \mathit{xs}^*)) \\
  \sigma := & \ \{\mathsf{EqLB} \mapsto \texttt{@Eq (List B)}, \ \ \mathsf{mAB} \mapsto \texttt{@map A B}, \\
            & \ \ \mathsf{rA} \mapsto \texttt{@reverse A}, \ \ \mathsf{rB} \mapsto \texttt{@reverse B}, \ \ f^* \mapsto \texttt{f}, \ \ \mathit{xs}^* \mapsto \texttt{xs} \\
            & \ \ \mathsf{LA} \to \texttt{List A}, \ \ \mathsf{LB} \to \texttt{List B}, \ \ \mathsf{A} \to \texttt{A}, \ \ \mathsf{B} \to \texttt{B}\} \\
\end{aligned}$$
where $\mathsf{EqLB} : \mathsf{LB} \to \mathsf{LB} \to \mathsf{Bool}, \
\mathsf{rA} : \mathsf{LA} \to \mathsf{LA}, \ \mathsf{rB} : \mathsf{LB} \to \mathsf{LB}, \ 
\mathsf{mAB} : (\mathsf{A} \to \mathsf{B}) \to \mathsf{LA} \to \mathsf{LB}, \
f^* : \mathsf{A} \to \mathsf{B}, \ \mathit{xs}^* : \mathsf{LA}$.

In a sense, the $\text{HOL}^*$ (type) instances are ``abstracted'' to $\text{HOL}^*$ (type)
variables. Note that the logical rules of $\text{HOL}^*$ are not relevant to this abstraction procedure, only
the term calculus $\lambda_\to^*$ is involved. Therefore, we name this procedure $\lambda_\to^*$ \textit{abstraction}.

However, $\lambda_\to^*$ abstraction is not directly applicable to \vmaprev and
\vrevrev, because dependent arguments of polymorphic functions occurring in them contain
universally quantified variables. Naturally, we would like to instantiate the quantifiers to make $\lambda_\to^*$
abstraction applicable.

\subsection{Quantifier Instantiation} \label{exinst}

To understand how quantifiers should be instantiated, we investigate how they would
be instantiated if we were to prove the goal manually. There are at least two ways we can proceed. We can
either first use \texttt{@map\_reverse A B} to swap the outer \texttt{reverse} with \texttt{map}, then
use \texttt{@reverse\_reverse A} to eliminate \texttt{reverse}; or, first use
\texttt{@map\_reverse A B} to swap the inner \texttt{reverse} with \texttt{map}, then
use \texttt{@reverse\_reverse B} to eliminate \texttt{reverse}. Notice how the dependent
arguments of a function $f$ in the instantiated hypotheses match the dependent
arguments of $f$ in the $\text{HOL}^*$ instances of $f$ in the goal.

Quantifier instantiation in Lean-auto's monomorphization procedure is based
on a matching procedure that reflects the above observation. Given a set of formulas $S$,
the matching procedure first computes the set $M$ of $\text{HOL}^*$ instances occurring
in $S$, then matches type-quantified formulas in $S$ with elements of $M$. For example,
given $S=$\texttt{\{@map\_reverse, @reverse\_reverse, neg\_goal\}}, the set $M$ will be
\texttt{\{@reverse A, @reverse B, @map A B, @Eq (List B)\}}, all of which collected from \vneggoal.
The matching procedure will preform the following matchings:

\begin{enumerate}
  \item \texttt{@Eq (List $\beta$)} in \vmaprev with \texttt{@Eq (List B)},
    which produces \texttt{fun $\alpha$ => @map\_reverse $\alpha$ B}
  \item \texttt{@map $\alpha$ $\beta$} in \vmaprev with \texttt{@map A B},
    which produces \texttt{map\_reverse A B}
  \item \texttt{@reverse $\alpha$} in \vmaprev with \texttt{@reverse A} and \texttt{@reverse B},
    which produces \texttt{@map\_reverse A} and \texttt{@map\_reverse B}
  \item \texttt{@reverse $\beta$} in \vmaprev with \texttt{@reverse A} and \texttt{@reverse B},
    which produces \texttt{fun $\alpha$ => @map\_reverse $\alpha$ A} and \texttt{fun $\alpha$ => @map\_reverse $\alpha$ B}
  \item \texttt{@Eq (List $\alpha$)} in \vrevrev with \texttt{@Eq (List B)},
    which produces \texttt{@reverse\_reverse B}
  \item \texttt{@reverse $\alpha$} in \vrevrev with \texttt{@reverse A} and \texttt{@reverse B},
    which produces \texttt{@reverse\_reverse A} and \texttt{@reverse\_reverse B}
\end{enumerate}

Since \texttt{@reverse\_reverse A}, \texttt{@reverse\_reverse B} and \texttt{@map\_reverse A B}
are present, the instances produced are already sufficient for proving the goal.
But generally speaking, newly generated hypothesis instances and $\text{HOL}^*$ instances
in them can still be matched with each other (and existing ones) to produce new useful results.
% TODO: Give a concrete example taken from my graduation thesis?
Hence, the monomorphization procedure in Lean-auto uses a saturation loop which
repeats the matching procedure until either no new instances can be produced or a prescribed
threshold is reached.

\subsection{Challenges related to Lean4} \label{exqdet}

\noindent \textbf{Dependent Arguments are Dynamic}

\begin{figure}
  \begin{CenteredBox}
    \begin{lstlisting}[style=leanHH]
|!@DFunLike.coe!| : {F : Type (max u_1 u_5)}
  → {α : outParam (Type u_1)} → {β : outParam (α → Type u_5)}
  → [self : DFunLike F α β] → F → (a : α) → β a

@DFunLike.coe (A₀ →+ B₀) A₀ (fun x => B₀) AddMonoidHom.instFunLike f₀ a 
    \end{lstlisting}
  \end{CenteredBox}
  \caption{The function \texttt{DFunLike.coe} from MathLib4 and an expression
  containing it}
  \label{dfun}
\end{figure}

  In $\lambda C$, whether an argument is dependent depends on how previous arguments
are instantiated. Consider the example shown in Figure \ref{dfun}. The return
type \texttt{$\beta$ a} depends on the last argument \texttt{a : $\alpha$} in the signature of
\texttt{DFunLike.coe}. However, when $\beta$ is instantiated with \texttt{fun x => B$_0$}, as in the
expression at the bottom of Figure \ref{dfun}, the return type \texttt{$\beta$ a} reduces to \texttt{B$_0$},
which no longer depends on the last argument. Our monomorphization procedure takes preceding arguments into
consideration when determining whether an argument is dependent.

\noindent \textbf{$\text{HOL}^*$ Instances are Dynamic}

  In $\lambda C$, whether an expression is an $\text{HOL}^*$ instance is context-dependent.
Consider the simple expression \texttt{@reverse = @reverse}, where \texttt{reverse}
is the same as in Figure \ref{leanlistexplicit}. Although \texttt{@reverse} is polymorphic,
it \textit{behaves like} an $\text{HOL}^*$ variable in \texttt{@reverse = @reverse}. More formally,
let
$$\begin{aligned}
\varphi &:= (f = f) \\
\sigma  &:= \{f \mapsto \texttt{@reverse}, \ \ \alpha \mapsto \texttt{(} \forall \ \texttt{\{}\alpha \ \beta \ : \ \texttt{Type\}}, \ 
  \texttt{List} \ \alpha \to \texttt{List} \ \beta \texttt{)}\}
\end{aligned}$$
where $f : \alpha$. Then, \texttt{@reverse = @reverse} is the image of the $\text{HOL}^*$ formula $\varphi$
under $\sigma$. Intuitively, the dependent arguments of \texttt{reverse} can be ``absorbed''
into the $\text{HOL}^*$ type variable $\alpha$ because both dependent arguments of \texttt{reverse} are unapplied.
Our monomorphization procedure is able to detect such context-dependent $\text{HOL}^*$ instances.

\noindent \textbf{Definitional Equality}

  As mentioned before, two syntactically different expressions can be definitionally
equal in Lean4. Theoretically speaking, reducing all expressions to their normal forms will solve
the definitional equality problem to a large extent. However, reduction is prohibitively
expensive on complex expressions in real-life Lean4 projects. Moreover, the reduced expression
could be much larger than the original expression, and might contain complex dependent types
that Lean-auto cannot handle. %TODO: Experiments

  Therefore, we have to find other methods to address definitional equality.
In Lean-auto, there are three separate occasions where definitional equality
has to be addressed.

  First, when a symbol is defined in Lean4, (potentially multiple) \textit{equational theorems} that
reflect the definitional equalities related to the symbol are automatically generated.
Lean-auto can be configured to collect equational theorems, perform reduction and
unfold constants, see Sect. \ref{sectprep}.

  Second, during $\lambda_\to^*$ abstraction, we would like $\text{HOL}^*$ instances
that are syntactically different but definitionally equal to be abstracted to the
same $\text{HOL}^*$ variable. Our $\lambda_\to^*$ abstraction algorithm keeps a
set $H$ of mutually definitionally unequal $\text{HOL}^*$ instances. Whenever a new $\text{HOL}^*$
instance $t$ is found, we test definitional equality of $t$ with elements of $H$
using \texttt{isDefEq}. Since \texttt{isDefEq} is still expensive,
a \textit{fingerprint} is computed for each $\text{HOL}^*$ instance, and fingerprint equality
is tested before calling \texttt{isDefEq}.

  Finally, even if two $\text{HOL}^*$ instances are definitionally unequal, there could still be nontrivial
relations between them. For example, suppose $f : \mathbb{N} \to \mathbb{N}$ is defined as
$f := \lambda (x : \mathbb{N}). g \ x \ x$, then $f$ is not definitionally equal to $g$. However, we do
have the nontrivial equation $\forall (x : \mathbb{N}). f \ x = g \ x \ x$. Lean-auto will attempt to generate such equations
after quantifier instantiation completes. For each pair of $\text{HOL}^*$ instances $c_1, c_2$,
Lean-auto tries to find terms $t_1, \dots, t_n$ such that $\forall x_1 \dots x_m. \ c_1 \ y_1 \ \dots \ y_l = c_2 \ t_1 \ \dots \ t_n$,
where $x_1, \dots, x_m$ are variables occurring in $t_1, \dots, t_n$, and
$\{y_1, \dots, y_l\}$ is a subset of $\{x_1, \dots, x_m\}$. For simplicity, this
definitional equality generation procedure is not included in Sect \ref{sectabst} and Sect \ref{sectinst}.

\noindent \textbf{Absorbing Typeclass Instance Arguments}

  In Lean4, many function's instance argument(s) are not dependent arguments,
for example the fourth argument of \texttt{HAdd.hAdd} mentioned in Sect. \ref{sectlean}.
Since instance arguments are usually large expressions synthesized by Lean4's typeclass
inference algorithm, their presence can result in large translation results.
Lean-auto's actual implementation attempts to absorbed typeclass arguments into
$\text{HOL}^*$ variables by instantiating typeclass instance quantifiers
and requiring $\text{HOL}^*$ instances to take typeclass arguments with them. For
simplicity, this detail is not included in Sect \ref{sectabst} and Sect \ref{sectinst}.
