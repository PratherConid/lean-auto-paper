\section{Preprocessing}\label{sectprep}

  Preprocessing translates Lean4 into dependent type theory, which involves
  handling various language features of Lean4. In this section, we list the major steps
  in Lean-auto's preprocessing.

% TODO: Collecting Inhabitation Facts

\noindent \textbf{Definitional Equality}

  To handle definitional equality in Lean4, Lean-auto performs reduction on the input
  expressions, using Lean4's builtin \texttt{Meta.transform} and \texttt{Meta.whnf}.
  This includes $\beta \zeta \eta \iota$ reduction and part of $\delta$
  reduction\footnote{$\delta$ reduction is constant unfolding, i.e. replacing a constant with its definition}.
  In Lean4, $\delta$ reduction is controlled by a reducibility setting,
  and Lean-auto allows users to specify the reducibility setting of preprocessing.
  For finer-grained control over which constants should be unfolded, Lean-auto allows
  users to supply an \textit{unfolding instruction} $u[f_1, \dots, f_n]$ and a
  \textit{definitional equality instruction} $d[g_1, \dots, g_n]$, where $f_i, g_i$
  are constants.

  For the definitional equality instruction, Lean-auto automatically collects all the definitional
  equalities associated with $g_1, \dots, g_n$ and combines them with the hypotheses
  supplied by the user. If the definition of $g_i$ involves pattern matching,
  there will be multiple definitional equalities associated with $g_i$, and Lean-auto
  will collect all of them.

  For the unfolding instruction, Lean-auto recursively unfolds $f_1, \dots, f_n$.
  To ensure termination, Lean-auto performs topological sort on $f_1, \dots, f_n$,
  where $f_i$ is sorted before $f_j$ if $f_j$ occurs in the definition of $f_i$. If
  $f_1, \dots, f_n$ cannot be topologically sorted, then there is cyclic dependency
  between them, and Lean-auto will fail in these cases.

\noindent \textbf{Inductive Types}

  Currently, Lean-auto supports polymorphic, nested and mutual inductive types
  when SMT solvers are used as the backend ATP. For other ATPs or unsupported
  inductive types, users can always manually supply the properties related
  to the inductive types as a workaround.

  The translation procedure for inductive types resembles monomorphization. For a polymorphic inductive type
  $T : \forall (\alpha_1 : \mathsf{U}_{\ell_1}) \ \dots \ (\alpha_n : \mathsf{U}_{\ell_n}). \mathsf{U}_\ell$,
  the translation attempts to find all relevant instances $T \ \alpha_1 \ \dots \ \alpha_n$,
  and translates each instance to a monomorphic inductive type in the SMT solver.
  For mutual and nested inductive types, the type of their constructors might contain
  other inductive types not occurring in the input expressions. These inductive
  types will be recursively collected and monomorphized by the translation procedure.

\noindent \textbf{Quantifier Introduction and Proof by Contradiction}

  To prepare for monomorphization, Lean-auto performs quantifier introduction on the goal and
  applies proof by contradiction. Suppose the goal is $\forall (x_1 : \alpha_1) \ \dots \ (x_n : \alpha_n). \beta$, 
  then quantifier introduction will introduce $x_1 : \alpha_1, \dots, x_n : \alpha_n$ to the context and replace
  the goal with $\beta$. Then, by applying proof by contradiction, the negation of
  the goal $h : \neg \beta$ is introduced into the context, and the goal is replaced
  with $\bot$.