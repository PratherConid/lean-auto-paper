\section{Quantifier Instantiation}\label{sectinst}
Given a context $\Gamma$ and a list of hypotheses $h_1 : t_1, \dots, h_n : t_n$, the quantifier
instantiation procedure of Lean-auto attempts to instantiate quantifiers in
$t_1, \dots, t_n$ to obtain terms suitable for $\lambda_\to^*$ abstraction.

First, we give a formal definition of \textit{instance}:

\begin{definition}
  Let $\Gamma$ be a $\lambda C$ context, and $t$ be a $\lambda C$ term which is type correct under $\Gamma$.
  \begin{enumerate}
    \item A constant instance of $t$ is a $\lambda C$ term of the
      form $\lambda (y_1 : s_1) \dots (y_m : s_m). t \ t_1 \ \dots \ t_k$ that is
      type correct under $\Gamma$, where $s_1, \dots, s_m, t_1, \dots t_k$ are $\lambda C$ terms.
    \item For $t = \forall (x_1 : r_1) \dots (x_n : r_n). b$, a hypothesis instance of
      $t$ is a $\lambda C$ term of the form $\forall (y_1 : s_1) \dots (y_m : s_m). b[t_1/x_1]\dots[t_n/x_n]$,
      where $s_1, \dots, s_m, t_1, \dots, t_n$ are $\lambda C$ terms, and the notion $t_1[t_2/x]$ stands
      for the term obtained by replacing all the $x$ in $t_1$ with $t_2$.
  \end{enumerate}
\end{definition}

Unless otherwise stated, when discussing instances of functions, we will always be
referring to \textit{constant instances}; when discussing instances of hypotheses, we will
always be referring to \textit{hypothesis instances}.

As mentioned in Sect. \ref{motex}, the quantifier instantiation procedure
is based on a saturation loop which matches $\text{HOL}^*$ instances of functions
with subterms of hypothesis instances. An instance of a function is called
an $\text{HOL}^*$ instance iff all of the function's dependent arguments are instantiated with terms
that do not contain bound variables. Formally, the set of all $\text{HOL}^*$ instances
in a $\lambda C$ term is defined as follows:

\begin{definition}
  Let $\Gamma$ be a $\lambda C$ context and $B$ be a set of variables, then
  \begin{enumerate}
    \item For variable $x$ and terms $t_1, \dots, t_n$,
      $$\mathsf{mholInsts}(\Gamma; B, x \ t_1 \ \dots \ t_n) := \left\{
        \begin{aligned}
          S \cup \{l\}, & & FV(l) \cap B = \emptyset \\
          S, & & \text{otherwise}
        \end{aligned}
      \right.$$
      where
      $$l := \mathsf{LFun}(\Gamma; x, (t_1 \ \dots \ t_n)) \ \ \ \ \ \ S := \bigcup_{t \in \mathsf{LArgs}(\Gamma; x, (t_1, \dots, t_n))} \mathsf{mholInsts}(\Gamma; V, t)$$
    \item For variable $x$ and terms $a, b$,
      \begin{align*}
        \mathsf{mholInsts}(\Gamma; B, \forall (x : a). b) = \mathsf{mholInsts}(\Gamma; B, \lambda (x : a). b) 
        \\ := \mathsf{mholInsts}(\Gamma; B, a) \cup \mathsf{mholInsts}(\Gamma, x : a; B \cup \{x\}, b)
      \end{align*}
    \item Otherwise, $\mathsf{mholInsts}(\Gamma; B, t) := \emptyset$
  \end{enumerate}
\end{definition}

\begin{algorithm}\label{matching}
  \DontPrintSemicolon
  \SetNoFillComment
  \SetKwFunction{matchFun}{\textsf{match}}
  \caption{Matching algorithm for quantifier instantiation}
  \Fn{\matchFun{$\Gamma; M, m, h$}}{
    \Input{$\lambda C$ context $\Gamma$, variable set $M$, and $\lambda C$ terms $m, h$}
    \Output{A set of unifiers}
    \Switch(\textbf{with}){h}{
      \Case(\tcc*[h]{Function application}){$a \ b$}{
        $\vmatches := \emptyset$ \;
        $f := \mathsf{getAppFn}(t)$ \;
        $\vargs := \mathsf{getAppArgs}(t)$ \;
        \For{$a : \vargs$}{$\vmatches := \mathsf{union}(\vmatches, \mathsf{match}(\Gamma; M, m, a))$}
        $\vlf := \mathsf{LFun}(\Gamma; f, arg)$ \;
        $\vmatches := \mathsf{union}(\vmatches, \mathsf{unify}(\Gamma; M, m, \vlf))$
      }
      \Case{$\forall (v : a). b$}{
        \Return $\mathsf{union}(\mathsf{match}(\Gamma; M, m, a), \mathsf{match}(\Gamma, v : a; M, m, b))$
      }
      \Case{$\lambda (v : a). b$}{
        \Return $\mathsf{union}(\mathsf{match}(\Gamma; M, m, a), \mathsf{match}(\Gamma, v : a; M, m, b))$
      }
      \Other{\Return $\emptyset$}
    }
  }
\end{algorithm}

The matching procedure in the saturation loop is handled by $\mathsf{matchInst}$ and $\mathsf{match}$.
\begin{enumerate}
  \item Given context $\Gamma$, variable set $M$ and terms $m, h$,
    $\mathsf{match}(\Gamma; M, m, h)$ returns all $M$-unifiers between term $m$ and the $\mathsf{LFun}$ of subterms of $h$.
    The pseudocode for $\mathsf{match}$ is given in \textbf{Algorithm \ref{matching}}. An auxiliary function
    $\mathsf{unify}$ is used in the pseudocode. Given $\lambda C$ context $\Gamma$, variable set $M$
    and two $\lambda C$ terms $t_1, t_2$, $\mathsf{unify}(\Gamma; M, t_1, t_2)$ returns a complete set of
    $M$-unifiers of $t_1$ and $t_2$ under $\Gamma$. In Lean, the \texttt{isDefEq} function can be used perform
    unification, but it is incomplete and returns at most one unifier.
  \item Given context $\Gamma$ and terms $m, h$,
    $\mathsf{matchInst}(\Gamma; m, h)$ computes all instances of the hypothesis $h$ which has some subterm whose
    $\mathsf{LFun}$ is $\beta\eta$-equivalent to $m$. To do this, $\mathsf{matchInst}$ introduces all leading non-prop $\forall$
    quantifiers into the context (as free variables), collects all the newly introduced free variables into a variable set $M$,
    then computes $\mathsf{match}(\Gamma'; M, m, h')$, where $\Gamma', h'$ are $\Gamma, h$ after introduction of free variables.
    For each unifier $(\Gamma, \Gamma', \sigma)$ in $\mathsf{match}(\Gamma'; M, m, h)$, $\mathsf{matchInst}$ computes $\overline{\sigma}(h)$,
    then abstracts newly introduced free variables in $\sigma$ as $\forall$ binders to generate an instance of $h$.
    $\mathsf{matchInst}(\Gamma; m, h)$ returns the set of instances of $h$ generated by this procedure. 
\end{enumerate}

\begin{algorithm}\label{saturate}
  \DontPrintSemicolon
  \SetNoFillComment
  \SetKwFunction{saturateFun}{\textsf{saturate}}
  \caption{Main saturation loop of quantifier instantiation}
  \Fn{\saturateFun{$\Gamma; H, \vmaxInsts$}}{
    \Input{$\lambda C$ context $\Gamma$, list of $\lambda C$ terms $H$, and threshold $\vmaxInsts$}
    \Output{A list of $\lambda C$ terms}
    $\vhi := H$ \tcc*[h]{List of hypothesis instances} \;
    $\vci := \mathsf{Queue.empty}()$ \tcc*[h]{A queue of constant instances} \;
    \For{h : H}{
      \For{$m : \mathsf{mholInsts}(\Gamma; \emptyset, h)$}{$\vci.\mathsf{push}(m)$}
    }
    $\vactiveci := \vci.\mathsf{copy}()$ \;
    \While{$! \ \vactiveci.\mathsf{empty}()$}{
      $m := \vactiveci.\mathsf{front}()$ \;
      $\vactiveci.\mathsf{popFront}()$ \;
      $\vprevhi = \vhi.\mathsf{copy}()$ \;
      \For{$h : \vprevhi$}{
        $\vnewhi := \mathsf{matchInst}(\Gamma; m, h)$ \;
        \For{$\vnh : \vnewhi$}{
          \lIf{$\vnh \in \vhi$}{\Continue}
          $\vhi.\mathsf{push}(\vnh)$ \;
          $\vnewci := \mathsf{mholInsts}(\Gamma; \emptyset, \vnh)$ \;
          \For{$\vnc : \vnewci$}{
            \lIf{$\vnc \in \vci$}{\Continue}
            $\vci.\mathsf{push}(\vnc)$ \;
            $\vactiveci.\mathsf{push}(\vnc)$
          }
        }
        \lIf{$\vhi.\mathsf{size}() + \vci.\mathsf{size}() > \vmaxInsts$}{\Break}
      }
      \lIf{$\vhi.\mathsf{size}() + \vci.\mathsf{size}() > \vmaxInsts$}{\Break}
    }
    $\vmonohi := \mathsf{List.empty}()$ \;
    \For{$h : \vhi$}{
      \lIf{$\mathsf{QMono}(\Gamma; \emptyset, h)$}{$\vmonohi.\mathsf{push}(h)$}
    }
    \Return $\vmonohi$
  }
\end{algorithm}

The saturation loop of quantifier instantiation is shown in \textbf{Algorithm \ref{saturate}}.
Given a $\lambda C$ context $\Gamma$ and a list $H$ of hypotheses, $\mathsf{saturate}$ returns
a list of instances of hypotheses in $H$ that are suitable for $\lambda_\to^*$ abstraction
(i.e. satisfy the $\mathsf{QMono}$ predicate).
Note that, in Lean-auto, when checking whether a hypothesis instance belongs
to a collection (e.g., set, list, queue, etc.) of hypothesis instances,
we test equality only up to \textit{hypothesis equivalence}.

\begin{definition}
For two $\lambda C$ terms $t_1, t_2$, $t_1$ and $t_2$ are equivalent as hypotheses
iff $t_1$ is a hypothesis instance of $t_2$ and $t_2$ is a hypothesis instance of $t_1$.
\end{definition}

Checking membership up to equivalence ensures that collections of hypothesis
instances in our algorithms are free of redundant entries. Note that equivalence testing
can be reduced to unification, which can in turn be approximated by \texttt{isDefEq}.
 